\addcontentsline{toc}{chapter}{Conclusion}
\chapter*{Conclusion}

L'objectif principal de mon travail était d'identifier et de caractériser les modifications du programme transcriptionnel associées à des mutations somatique des TFBS dans les léiomyosarcomes. Un pipeline d'analyses de données a été réalisé afin d'extraire les mutations somatiques des TFBS, il a pu être lancé sur les 80 échantillons des 68 cas actuellement disponibles. 

Il ressort qu'en moyenne, chaque échantillon possède 50 mutations différentes des TFBS et ce quelle que soit la localisation de la tumeur. Pour des raisons techniques et logistiques, seulement trois de ces mutations, chacune appartenant à un léiomyosarcome différent, ont pu être testées. Il faudra donc étendre les vérifications à un plus grand nombre de mutations et si possible à tous les échantillons pour juger de la viabilité des résultats. 

Dans les échantillons, j'ai pu observer que les mutations des TFBS sont très peu nombreuses (0.05\%) ce qui correspond aux observations de \citeauthor{Melton}. 92 gènes cibles des facteurs de transcription POU2F1,MEF2 et FOXJ2 associés à des mutations des TFBS présentent des changements d'expression. Ces différents gènes modifie le programme transcriptionnel en jouant sur la fixation de l'ADN, des protéines, de l'ARN polymérase II ou encore dans la différenciation neuronale. 

De telles modifications pourraient être à l'origine du développement de tumeurs. Les gènes n'étant pas mutés, il faudra réaliser des analyses \og Chip Seq\fg afin de savoir si le facteur de transcription associé au site muté s'y fixe. Dans le cas d'une non fixation au niveau de la tumeur, l'hypothèse sera que la modification d'expression du gène est lié à la mutation du TFBS. 

L'équipe \textit{Génétique et biologie des sarcomes} dirigée par Frédéric Chibon devrait débuter la seconde phase du projet \og Soft tissue cancer\fg ~cette année, en complétant le jeu de données afin d'obtenir une cohorte d'au moins 200 cas. Le pipeline d'analyse pourra être exécuté sur ces nouveaux cas afin de confirmer ou d'infirmer les observations précédentes. De même, les mutations détectées pourront par la suite être validées ou invalidées par des analyses biologiques.

Les test statistiques réalisés lors de mon projet ont été réalisés sur un faible nombre d'échantillons. Il faudra donc les confirmer ou les infirmer en utilisant les nouveaux cas issus de la seconde phase du projet. Pour obtenir des résultats optimaux, il faudra avoir une répartition équitable des échantillons pour chaque localisation.

Les bases de données des gènes cibles utilisées ne contiennent que des données vérifiées biologiquement. Pour compléter le nombre de gènes cibles, il faudra utiliser des bases de données prédictives.

Les gènes récupérés autour des TFBS mutés peuvent être des oncogènes ou des suppresseurs de tumeurs déjà connus. Une analyse de ces gènes pourra donc permettre de ressortir des éléments déjà observés dans la littérature et ainsi orienter les futures analyses sur les éléments jusqu'alors inconnus.




