\newglossaryentry{sarcomes}{
	name={Sarcomes},
    text={sarcomes},
	description={: cancer très rare se développant à partir de cellules conjonctives.}}
    
\newglossaryentry{cellules conjonctives}{
	name={Cellules conjonctives},
    text={cellules conjonctives},
	description={: cellules qui composent les tissus conjonctifs, elles sont séparées par une matrice extra-cellulaire}}
    
\newglossaryentry{mutations somatiques}{
	name={Mutations somatiques},
    text={mutations somatiques},
	description={: mutations qui sont présentes uniquement dans les cellules tumorales}}

\newglossaryentry{modifications epigenetiques}{
	name={Modifications épigénétiques},
    text={modifications épigénétiques},
	description={: modifications qui changent l'expression des gènes mais ne sont pas liées à une mutation de la séquence d'ADN}}
    
\newglossaryentry{sequences regulatrices}{
	name={Séquences régulatrices},
    text={séquences régulatrices},
	description={: régulent l'activité des gènes en activant ou inhibant les promoteurs de ceux ci}}   
    
\newglossaryentry{oncogenese}{
	name={Oncogénèse},
    text={oncogénèse},
	description={: ensemble des mécanismes induisant la formation de tumeurs}}    

\newglossaryentry{facteur de transcription}{
	name={Facteur de transcription},
    text={facteur de transcription},
	description={: protéine qui régule l'expression des gènes}}
    
\newglossaryentry{transcription}{
	name={Transcription},
    text={transcription},
	description={: processus par lequel le code génétique contenu dans l'ADN peut être transmis}}    

\newglossaryentry{sequence nucleotidique}{
	name={Séquence nucléotidique},
    text={séquence nucléotidique},
	description={: assemblage des bases azotées Adénine, Guanine, Thymine et Cytosine}} 

\newglossaryentry{genome de reference}{
	name={Génome de référence},
    text={génome de référence},
	description={: génome représentatif d'une espèce donnée}}
    
\newglossaryentry{ADN constitutionnel}{
	name={ADN constitutionnel},
    text={ADN constitutionnel},
	description={: qui est présent dans les cellules de l'organisme}}
    
\newglossaryentry{ADN tumoral}{
	name={ADN tumoral},
    text={ADN tumoral},
	description={: qui n'est présent que dans les cellules tumorales}}

\newglossaryentry{sequences dupliquees}{
	name={Séquences dupliquées},
    text={séquences dupliquées},
	description={: qui ont les mêmes positions de départ et de fin}}
    
\newglossaryentry{read group}{
	name={Read group},
    text={read group},
	description={: nomenclature constituée au minimum du nom de l'échantillon, de la plateforme de séquençage et du lieu de séquençage }}

\newglossaryentry{ordre caryotypique}{
	name={Ordre caryotypique},
    text={ordre caryotypique},
	description={: du chromosome 1 au 22 suivit des chromosomes X, Y et M }}
    
\newglossaryentry{lignee germinale}{
	name={Lignée germinale},
    text={lignée germinale},
	description={: comprend les cellules à l'origine des gamètes }}

\newglossaryentry{copy number}{
	name={Copy number},
    text={copy number},
	description={: nombre de copies d'un gène }}

\newglossaryentry{machine learning}{
	name={Machine learning},
    text={machine learning},
	description={: mise en place d’algorithmes en vue d’obtenir une analyse prédictive à partir de données }}

\newglossaryentry{frequence allelique}{
	name={Fréquence allélique},
    text={fréquence allélique},
	description={: nombre de séquences couvertes par un élément sur le nombre de séquences totales }}

\newglossaryentry{faux positifs}{
	name={Faux positifs},
    text={faux positifs},
	description={: mutations détectées comme somatique alors qu'elles se trouvent dans l'échantillon constitutionnel }}
    
\newglossaryentry{chromatogramme}{
	name={Chromatogramme},
    text={chromatogramme},
	description={: représentation des signaux de fluorescence pour une séquence}}

\newglossaryentry{homozygote}{
	name={Homozygote},
    text={homozygote},
	description={: porté par les deux allèles d'un gène}}

\newglossaryentry{heterozygote}{
	name={Hétérozygote},
    text={hétérozygote},
	description={: porté par un seul des deux allèles d'un gène}}
    
\newacronym{icgc}{ICGC}{International Cancer Genome Consortium} 
\newacronym{lms}{LMS}{Léiomyosarcomes}
\newacronym{stc}{STC}{Soft Tissue Cancer}
\newacronym{tfbs}{TFBS}{Transcription Factor Binding Site}
\newacronym{ngs}{NGS}{Next Generation Sequencing}
\newacronym{adn}{ADN}{Acide DésoxyriboNucléique}
\newacronym{arn}{ARN}{Acide RiboNucléique}
\newacronym{snp}{SNP}{Single Nucleotide Polymorphism}
\newacronym{indel}{INDEL}{INsertion/DELétion}
\newacronym{sam}{SAM}{Sequence Alignment/Map}
\newacronym{bam}{BAM}{Binary Alignment/Map}
\newacronym{vcf}{VCF}{Variant Calling Format}
\newacronym{bed}{BED}{Browser Extensible Data}
\newacronym{gff}{GFF}{General Feature Format}
\newacronym{cram}{CRAM}{Compress Reference-based Alignment/Map}
\newacronym{loh}{LOH}{Loss of Heterozygocity}
\newacronym{pwm}{PWM}{Position Weight Matrix}
\newacronym{mcia}{MCIA}{Mésocentre de Calcul Intensif Aquitain}
\newacronym{fpkm}{FPKM}{Fragment Per Kilobase Of Exon Per Millon Fragments Mapped}