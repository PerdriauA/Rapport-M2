\appendix
\titleformat{\chapter}[frame]
  {\huge\normalfont\sc} 
  {\filright\footnotesize\enspace ANNEXE \thechapter\enspace}  {8pt}
  {\filcenter} 
  
\chapter{Venn}

\centering
\begin{tikzpicture}[>=stealth,on grid,auto, text centered]
	\tikzset{venn circle/.style={draw,circle,minimum width=8cm,fill=#1,opacity=0.5}}

    \begin{scope}
     \node [venn circle = red] (A) at (0,0) {$Varscan : 108$};
  	 %\node [venn circle = blue] (B) at (60:6cm) {$Samtools : 82$};
  	 \node [venn circle = green] (C) at (0:6cm) {$MuTect : 188$};
     %\node[left] at (barycentric cs:A=1/2,B=1/2 ) {$31$}; 
     \node[below] at (barycentric cs:A=1/2,C=1/2 ) {$83$};   
  	 %\node[right] at (barycentric cs:B=1/2,C=1/2 ) {$49$};   
  	 %\node[below] at (barycentric cs:A=1/3,B=1/3,C=1/3 ){$30$};
    \end{scope}
\end{tikzpicture}
Détection des variants pour le LMS1 (intra-abdominal)
\begin{itemize}
\item 75 \% de Varscan est commun avec MuTect
\item 
\end{itemize}