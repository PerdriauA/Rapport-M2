\addcontentsline{toc}{chapter}{Introduction}
\chapter*{Introduction}

Les cancers représentent la première cause de mortalité en France \citep{Insee} et la cinquième dans le monde \citep{OMS}. Ils se développent à partir de cellules anormales qui se multiplient de manière incontrôlée au détriment du reste de l'organisme. Les études menées ces 40 dernières années ont permis de démontrer que des mutations génétiques pouvaient être à l'origine des cancers \citep{Cancer}. Cette question est toujours particulièrement étudiée dans l'ensemble des cancers dont les \gls{sarcomes}.

Les sarcomes sont des tumeurs cancéreuses rares qui se développent à partir de \gls{cellules conjonctives}. On les classe en plusieurs catégories en fonction des tissus qu'ils touchent : osseux, mous et certains viscères. Parmi les sarcomes des tissus mous les plus fréquents, les léiomyosarcomes (\acrshort{lms}) sont des cancers qui se développent à partir de cellules musculaires lisses.

L'\acrlong{icgc} (\textit{\acrshort{icgc}}) a pour objectif de construire un catalogue des anomalies génétiques (\gls{mutations somatiques}, \gls{modifications epigenetiques}, etc.) des tumeurs provenant de 50 types ou sous-types de cancer ayant une importance clinique et sociale mondiale \citep{ICGC}. L'ICGC finance 78 projets à travers le monde parmi lesquels se trouve le projet \og \acrlong{stc}\fg ~(\textit{\acrshort{stc}}) porté par l'équipe \textit{Génétique et biologie des sarcomes} (GBS) dirigée par Frédéric Chibon au sein de l'unité \textit{Inserm U1218} et de l'institut \textit{Bergonié}. Le but de ce projet unique au monde est d'améliorer la connaissance de la génétique des LMS, représentant 5 à 10\% des sarcomes des tissus mous, en étudiant les altérations génétiaques et structurales.

Dans une publication de juillet 2015, \citeauthor{Melton} démontrent l'importance que peuvent avoir les mutations des \gls{sequences regulatrices} sur l'expression des gènes dans l'\gls{oncogenese}. Ces séquences sont de mieux en mieux décrites notamment via des projet comme ENCODE (Enclyclopedia Of DNA Elements) \citep{encode}, ce qui permet de les isoler du génome complet. Il est alors possible d'identifier de potentielles variations génétiques. Ces séquences altérées pourraient alors avoir un rôle oncogène au sein du programme transcriptionnel tumoral. 

L'objectif principal de ce stage est d'identifier et de caractériser les mutations dites somatiques des sites de fixation des facteurs de transcription (\acrshort{tfbs}) dans les LMS, c'est-à-dire, qui sont uniquement dans les cellules tumorales. Dans cette optique, une étude bio-informatique va être réalisée sur des données issues de la technique \og \acrlong{ngs} \fg ~(\acrshort{ngs}) d'échantillons d'\acrlong{adn} (\acrshort{adn}) tumoraux et constitutionnels d'une série de 68 LMS annotés. 




